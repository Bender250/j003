\documentclass[a4paper]{article}
\usepackage[english, czech]{babel}
\usepackage[T1]{fontenc}
\usepackage{lmodern}
\usepackage{amssymb}
\usepackage[utf8]{inputenc}
\usepackage{graphicx}

\usepackage{amsthm}
\usepackage{amsmath}
\usepackage{color}
\usepackage{microtype}
\usepackage{url}
\usepackage{enumerate}
\title{J003 -- Fundamental Concepts of Computer Science and Some Surprising Discoveries: 2. sada}
\author{Karel Kubíček}
\date{\today}
\begin{document}
\maketitle

\section*{Úloha 1}
K důkazu použijeme totálně vyčíslitelnou funkci, která provede redukci slov z jazyka $L_{H}$ do slov jazyka $L_{U}$.

\[
    f(w) = 
    \begin{cases}
        \langle \mathcal{T}_{\mathrm{inf}} \rangle, & \text{jestliže } w \text{ není kódem žádné dvojice skládající se z TS a slova, } \\
        \langle \mathcal{T}_{\mathcal{M}, u} \rangle, & \text{jestliže } w = \langle \mathcal{M}, u \rangle \text{ pro nějaký TS } \mathcal{M} \text{ a nějaké slovo } u,
    \end{cases}
\]

Kde $\mathcal{T}_{\mathrm{inf}}$ je TS, který cyklí. Zato $\mathcal{T}_{\mathcal{M}, u}$ je TS, který simuluje TS $M$ na slově $u$ a to vždy akceptuje.

Funkce $f$ je totálně vyčíslitelná, jedná se tedy o Turingovu redukci. Musíme ještě dokázat, že redukce zachovává příslušnost slov v jazyce.

\uv{Pokud $w \in L_{H} \implies f(w) \in L_{U}$}: jelikož $w \in L_{H}$, pak jistě simulovaný stroj zastaví, a tedy je slovo akceptováno.

\uv{Pokud $w \not \in L_{H} \implies f(w) \not \in L_{U}$}: to, že $w \not \in L_{H}$ znamená, že simulovaný stroj cyklí (buď $w$ není kódem dvojice TM a slova, pak cyklí strom $\mathcal{T}_{\mathrm{inf}}$, nebo je $w = \mathcal{T}_{\mathcal{M}, u}$, a tedy $\mathcal{T}$ cyklí nad $u$). Pak výsledný stroj nedojde do akceptujícího stavu, a tedy $f(w) \not\in L_{U}$.

\section*{Úloha 2}

TODO

\section*{Úloha 3}

\begin{enumerate}[(a)]
    \item Řešení je obdobné, jako u úlohy 1, jen v redukci předáme slovo $001$ tak, aby se simulovaný stroj pouštěl vždy na něm.
    \item Redukce funguje následovně.
    
    \[
        f(w) = 
        \begin{cases}
            \langle \mathcal{T}_{\emptyset} \# \mathcal{T}_{\Sigma^{*}} \rangle, & \text{jestliže } w \text{ není kódem žádné dvojice skládající se z TS a slova, } \\
            \langle \mathcal{T}_{u} \# (\mathcal{T}_{u} \cap M) \rangle, & \text{jestliže } w = \langle \mathcal{M}, u \rangle \text{ pro nějaký TS } \mathcal{M} \text{ a nějaké slovo } u,
        \end{cases}
    \]
    
    Kde první případ generuje TS akceptující prázdný jazyk a TS akceptující všechna slova, které jsou odlišné, a tedy je takový vstup zamítnut.
    
    V druhém případě k zadané dvojici $M, u$ vytvoříme v redukci stroj $\mathcal{T}_{u}$, který akceptuje pouze slovo $u$. Vstupem pro TS jazyka $L_{EQ}$ je $\mathcal{T}_{u}$, oddělovač \# a průnik $\mathcal{T}_{u} \cap M$, přičemž průnik lze snadno realizovat pomocí spuštění obou TS a akceptováním pouze v případě, že oba TS skončily v akceptujícím stavu.
\end{enumerate}

\section*{Úloha 5}

Důkaz provedeme redukováním $L_{U} \leq_{R} L_{all}$, přičemž víme, že $L_{U} \not \in \mathcal{L}_{R}$.

Redukční funkce bude pro vstup $u = \textrm{Kod}(M)\#w$ následovný Turingův stroj (pokud $u$ nekóduje TS \# a slovo, pak vrací TS prázdného jazyka).

TS se podívá na vstupní slovo $v$. Pokud $v \not = u$, pak jej akceptuje. Pokud $v = u$, pak spustí $M$ na $v$.


\section*{Úloha 7}

Tvrzení dokážeme konstrukcí daného TS. V obou případech se bude jednat o třípáskové stroje, kde první páska obsahuje vstup a na dalších 2 probíhá výpočet.

\begin{enumerate}[(a)]
    \item První páska obsahuje vstup, po každé iteraci se na této pásce posouvá čtecí hlava o jedno pole dále a pokud narazí na pravou zarážku, tak ukončí výpočet a výstup překopíruje z aktuální pásky na první pracovní pásku.

    Zbylé pásky před začátkem výpočtu obsahují po 1 symbolu 0. Iterace překopíruje za konec jedné pásky tolik 0, kolik je na druhé pásce. V další iteraci se zapisuje na druhou pásku.

    Cena výpočtu je v $\mathcal{O}(2^{n} \cdot 4)$. Násobek 4 je dán nutností přesunu z konce pásky na začátek před započtením kopírování a závěrečným překopírováním výstupu. To je v $\mathcal{O}(e(n))$.

    \item Výpočet fibonacciho čísel probíhá obdobně jako výpočet mocniny 2. Změna je v tom, že na začátku inicializujeme jednu pásku na 0 a druhou na 1.
\end{enumerate}

\end{document}





